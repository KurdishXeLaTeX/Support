\documentclass{article}

\usepackage[bidi=basic]{babel}

\babelprovide[import, main, maparabic]{centralkurdish}
\babelprovide[onchar=ids fonts]{polytonicgreek}
\babelprovide[onchar=ids fonts]{english}

\babelfont{rm}{Yas}
\babelfont[polytonicgreek, english]{rm}{Times New Roman}

\title{بەڵگەیەک بۆ نموونە}
\author{نێوی من}
\date{\today}

\begin{document}


\maketitle

\section{\TeX چییە؟}

\TeX{} بۆ یەکەم جار لە ساڵی ١٩٧٨ لە لایەنی زانا و بلیمەتی کۆمپیوتەر
دۆناڵد کنووس (Donald Knuth) ەوە ساز کرا. \TeX{} لە سەر ئەو بۆچوونەدا
پێکهاتووە کە نووسەر دەبێ کاری نووسین بکا نەک ڕازاندنەوەی دەق. واتە، بۆ
نووسەری بەڵگەیەک دەبێ ئەو ئامێرانە دابین کرابێ کە بتوانێ بە هەموو
پێداویستییەکانی لە نووسیندا بگا، بگرە بۆ نووسینی فرمووڵێکی بیرکاری بێ
یان هاوکێشەیەکی کیمیایی یان پارچە شیعرێک. بۆیەش، لە نێو ئاکادیمیا و ئەو
کەسانه کە بە لێهاتوویی لە سەر بابەتێکی زانستی دەنووسن لە \TeX{} زۆر بە
بەربڵاوی کەڵک وەردەگیردرێ. وشەی \TeX{} لە وشەی یۆنانی \textit{τέχνη}
یەوە دێ کە بە مانای هونەرە. لە بەر ئەوەی فۆنیمی χ لە کوردی و زۆر زماندا
بەرانبەرێکی نییە، دەتوانین لە کوردی بڵەین \textit{تێک} یان
\textit{تێخ}.

\paragraph{تایبەتمەندییەکانی \TeX} بە کورتی، لە تایبەتیمەندییە هەرە
گرینگەکانی تێک ئەمانەن:
 
\begin{itemize}
\item نووسینی گۆڤار، کتێب، ڕاپۆرتی تەکنیکی و سڵایدی ئامادەکاری
\item بەڕێوە بردنی بەڵگە گەورەکان کە زۆر بەش و بەند و سەرچاوەیان تێدایە
\item نووسینەوەی فرمووڵە پڕ وردەکارییەکان لە بیرکاریدا
\item سازکردنی ئۆتۆماتیکی سەرچاوەکان
\item بە ئاسانی کەڵک وەرگرتن لە وێنە و دیاگرام
\item گۆڕانەوە بە شێوەکانی دیکەی نیشان دانی دەق وەکوو HTML
\end{itemize}


\begin{figure}
\caption{نیشان}
\end{figure}

\begin{figure}
\caption{نیشان}
\end{figure}
\begin{figure}
\caption{نیشان}
\end{figure}
\begin{figure}
\caption{نیشان}
\end{figure}
\begin{figure}
\caption{نیشان}
\end{figure}
\begin{figure}
\caption{نیشان}
\end{figure}
\begin{table}
\caption{نیشان}
\end{table}

\end{document}